\documentclass[a4, fontsize=10pt]{scrartcl}

\usepackage{graphicx}
%\usepackage{mathptmx}      % use Times fonts if available on your TeX system
\usepackage{multicol}
\usepackage{multirow}

% Eigene Defs
\usepackage{amsmath,amsfonts,amssymb}
\usepackage{xspace}
\usepackage{intmacros}
\usepackage{subfigure}
\usepackage[algoruled]{algorithm2e}
\usepackage{tikz}
\newcommand*\circled[1]{%
  \tikz[baseline=(C.base)]\node[draw,circle,inner sep=0.3pt](C){#1};
  \hspace{-1.3ex}
}
\newcommand{\iv}[2]{\ensuremath{[#1, #2]}\xspace}
\newcommand{\aff}[1]{\ensuremath{\hat{#1}}\xspace}
\def\wlst{\ensuremath{\mathcal{L}}\xspace}
\def\flst{\ensuremath{\mathcal{L_\mathrm{final}}}\xspace}
\def\den{\ensuremath{\bfm{\Phi}}\xspace}
\newcommand{\pow}[0]{\ensuremath{\mathrm{pow}}\xspace}
%\newcommand{\rad}[0]{\ensuremath{\mathrm{pow}}\xspace}
\graphicspath{{./figures/}}

\newcommand{\yalaa}{\texttt{YalAA}\xspace}

\pagestyle{empty}

\title{\yalaa - Yet Another Library for Affine Arithmetic}
\author{Stefan Kiel}

\setlength{\textheight}{24.5cm}

\begin{document}
\begin{center}
{\Large\bf
% Begin Title REQUIRED
\yalaa\ - Yet Another Library for Affine Arithmetic
% End Title
}

\bigskip

{\large
% Begin Author's Name REQUIRED
Stefan Kiel\\
% End Name
}
\medskip
% Begin Affiliation REQUIRED
University of Duisburg-Essen
% End Affiliation

% % Begin Address REQUIRED
% Ecole Normale Sup\'erieure de Lyon\\
% 69007 Lyon, France\\
% % End Address

% % Begin E-Mail OPTIONAL, REQUIRED for the corresponding author
% scan2010@ens-lyon.fr
% % End E-Mail
\end{center}
\medskip

% Begin Abstract REQUIRED
%\begin{abstract}
Affine arithmetic (AA) is a model for verified computations proposed by Comba
and Stolfi [1]. In contrast to interval arithmetic it tracks first-order
correlations in variables during the computation and thus often leads to
tighter range bounds for functions suffering from the dependency effect. An
affine form $\aff x = x_0 + \sum_{i=1}^n \x_i\epsilon_i$ has a central value
$x_0$ and error terms consisting of a partial deviation $x_i$ and a symbolic
noise variable $\epsilon_i \in \iv{-1}{1}$ modeling linear correlations.
Several improvements for the original model were proposed in [2]-[5]. However,
existing publicly available implementations (e.g. \texttt{libaa},
\texttt{libaffa}) have a number of shortcomings and do not support these
extensions.

% AA has attracted some attention in the last years even from outside of the
% classical interval community (...). 
 % The reference implementation libaa from the inventors of affine
% arithmetic is a non object oriented C library, which does not fit well into
% modern C++ software and further lacks implementations of several elementary
% functions like $\cos, \sin, \tan$.  Another freely available implementation
% named libaffa is object oriented and supports these function, but its
% implementation is not fully verified and further it is not very flexible.

In this talk we present \yalaa, a newly developed object-oriented library for
AA. Similarly to the \texttt{Boost.Interval} package, it uses  a configurable
base type for representation of the partial deviations. \yalaa's functionality is
controlled by policy classes: \texttt{ErrorTerm}, \texttt{AffineCombination},
\texttt{ArithmeticKernel}, \texttt{ErrorPolicy}, \texttt{AffinePolicy}.

\texttt{ErrorTerm} defines the representation of the symbolic noise variables
and the partial deviations, while \texttt{AffineCombination} models a
combination of several \texttt{ErrorTerm} objects and supplies the basic
affine operations: addition, scaling and translation.  The kernel implements
the actual mathematical operations. The library also contains a kernel
suitable for standard floating-point types. If possible it implements the
elementary functions as described in [6]. Otherwise we use a Chebyshev
interpolation based approach. We plan to implement other kernels featuring the
faster computation model proposed in a recent SCAN talk [3]. Handling such errors
as domain violation or overflows is controlled by the
\texttt{ErrorPolicy}. \texttt{AffinePolicy} controls the way new affine noise
symbols are introduced. This allows us to implement the AF1 and AF2 forms
described in [2].

Currently we are working on an implementation of generalized interval
arithmetic [7] in \yalaa.  Similarly to AA, it tracks first order
correlations but represents the partial deviations with tight intervals. A
further goal is the support for higher order forms like those introduced in
[4],[5]. However, a new arithmetic kernel which exploits the higher order
noise symbols in the approximation of non-affine functions would be necessary
in this case. Note that this topic is not entirely explored because the relevant
publications focus on polynomial functions. Consequently they do not
cover non-affine operations or elementary
functions other than multiplication or the integer power function.\\
%\end{abstract}

\small{
{\bf References:}
\begin{description}
\item[{[1]}] J. L. D. Comba and J. Stolfi \textit{Affine Arithmetic and its Application to
  Computer Graphics}.
\item[{[2]}] F. Messine \textit{Extensions of Affine Arithmetic:
    Applications to Unconstrained Global Optimization}.
\item[{[3]}] J. Ninin and F. Messine \textit{Reliable Affine Arithmetics}.
\item[{[4]}] G. Bilotta \textit{Self-Verified Extension of Affine Arithmetic
    to Arbitrary Order}. 
\item[{[5]}] F. Messine and A. Touhami \textit{A General Reliable Quadratic Form: An Extension of Affine Arithmetic}.
\item[{[6]}] L. H. de Figueiredo and J. Stolfi \textit{Self-Validated Numerical Methods and Applications}.
\item[{[7]}] E. Hansen \textit{A Generalized Interval Arithmetic}.
%\item[{[7]}] G. Bilotta \textit{Self-Verified Extension of Affine Arithmetic to Arbitrary Order}.
\end{description}
}
\end{document}
